% vim: tw=82
\chapter{Basic User Interaction}

KnightOS has some basic ideas around how you interact with the system and the
programs running on that system. Thought KnightOS is highly customizable, these
ideas persist through most KnightOS installations, and they certainly apply to the
base installation.

\section{Special Keys}

There are a number of keys that have special meaning on KnightOS. Programs are
able to override these meanings, but they generally hold true.

\begin{description}
    \item[Function Keys] \hfill \\
        The \textbf{F1}-\textbf{F5} keys are the uppermost keys on your
        calculator's keypad. They are labelled with their function number in green
        near the key. These keys have ``meta'' uses in KnightOS, and are used to
        interact with the system and programs on it.
    \item[Function 1] \hfill \\
        The \textbf{F1} key is used to return to the program launcher from other
        programs.  The active program will be suspended and you can select
        another.
    \item[Function 2-4] \hfill \\
        The three middle function keys have context-specific actions. These
        interact with whichever program you have open. Consult the documentation
        for each program to learn more.
    \item[Function 5] \hfill \\
        The \textbf{F5} key is used to switch between programs. It will take you
        to a list of running programs to choose from.
    \item[CLEAR] \hfill \\
        The \textbf{CLEAR} is often used to return to the previous screen. It
        works as a ``Back'' button, of sorts. It is also used as ``Backspace''
        when editing text.
    \item[MODE] \hfill  \\
        The ``MODE'' key will exit the current program.
    \item[2ND, ENTER] \hfill \\
        The \textbf{2ND} and \textbf{ENTER} keys generally serve the same purpose
        - committing to a selection. The \textbf{2ND} key is included in this
        action because it is more natural to use it when also using the arrow
        keys.
    \item[Arrow Keys] \hfill \\
        The arrow keys are used to interact with menus and many screens on
        KnightOS. You can also use it, when editing text, to move your cursor.
    \item[ON] \hfill \\
        The \textbf{ON} key, in addition to starting the calculator, can be used
        for a number of ``meta'' uses. This gives you the ability to deal with
        troublesome situations, including killing a misbehaving program and
        running an immediate reboot. There is more information on this in section
        TODO.
\end{description}

\section{Starting and Exiting Programs}

This section still needs to be written.

\section{Switching Between Programs}

This section still needs to be written.
