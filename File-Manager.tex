% vim: tw=82
\chapter{File Manager}

One of the things that makes KnightOS special is that it uses a proper file
system. You can organize your files into directories and subdirectories just like
you can on your computer. The KnightOS directory layout was inspired by Unix,
which will be familiar to many users. To manage this file system, you should use
the File Manager program.

\section{File System Layout}

The KnightOS file system can be compared to that of Unix-like systems. As a user,
you will probably only care about the ``home'' directory, which is selected by
default when you open the file manager. However, we'll offer a brief overview of
the purpose of the other directories here:

\begin{itemize}
    \item ``/'' is the root of the filesystem. Everything is a subdirectory of it.
    \item ``/bin'' contains your installed programs. Don't mess with it.
    \item ``/etc'' contains your settings and other miscellanous files.
    \item ``/home'' is where all of your personal files should be stored.
    \item ``/include'' contains headers for programmers.
    \item ``/lib'' contains libraries for programmers. Don't mess with it.
    \item ``/share'' contains resources for programs, like icons.
    \item ``/var'' contains data, like saved games and equations.
\end{itemize}

You shouldn't mess with any files you don't understand, but you
\textit{especially} shouldn't mess with the directories we told you not to.

\section{Navigating the File System}

This section needs to be written.

\section{Creating New Directories}

This section needs to be written.

\section{Moving, Renaming, and Deleting Files}

This section needs to be written.

\section{USB Connectivity}

This section needs to be written.
